% ---------------------------------------------------
% ----- Main document of the template
% ----- for Bachelor-, Master thesis and class papers
% ---------------------------------------------------
%  Created by Claudia Müller-Birn on 2012-08-17. (updated on 2013-04-03)
%  Freie Universität Berlin, Institute of Computer Science, Human Centered Computing (HCC). 
%
\documentclass[pdftex,a4paper,12pt]{scrartcl}   
%
%---------------------------------------------------
%----- Packages
%---------------------------------------------------
%
\usepackage[T1]{fontenc} 
\usepackage[utf8]{inputenc}
\usepackage[ngerman]{babel} %\usepackage[english]{babel}  
\usepackage{ae} 
\usepackage[utf8]{inputenc} % Allow Umlauts in input
\usepackage[numbers]{natbib}   

\usepackage{fancyref} 
\usepackage{fancyhdr} % Define simple headings 
\usepackage{xcolor}
\usepackage{url}
%
\usepackage[pdftex]{graphicx}  
\usepackage{hyperref} % turn all your internal references into hyperlinks
%\usepackage[pdfstartview=FitH,pdftitle={<<Titel der Arbeit>>}, pdfauthor={<<Autor>>}, pdfkeywords={<<Schlüsselwörter>>}, pdfsubject={<<Titel der Arbeit>>}, colorlinks=true, linkcolor=black, citecolor=black, urlcolor=black, hypertexnames=false, bookmarksnumbered=true, bookmarksopen=true, pdfborder = {0 0 0}]{hyperref}
% 
% a new command is defined that allows to include an empty page when needed
\newcommand{\blankpage}{
\newpage
\thispagestyle{empty}
\mbox{}
\newpage
}
%
%---------------------------------------------------
%----- PDF and document setup
%---------------------------------------------------
%
\hypersetup{
	pdftitle={Untersuchung von RRT* bei Autos},  % please, add the title of your thesis
    pdfauthor={Bernd Sahre},   % please, add your name
    pdfsubject={<<Bachelor thesis>, Institute of Computer Science, Freie Universität Berlin>}, % please, select the type of this document
    pdfstartview={FitH},    % fits the width of the page to the window
    pdfnewwindow=true, 		% links in new window
    colorlinks=false,  		% false: boxed links; true: colored links
    linkcolor=red,          % color of internal links
    citecolor=green,        % color of links to bibliography
    filecolor=magenta,      % color of file links
    urlcolor=cyan           % color of external links
}
%
%---------------------------------------------------      
%----- Settings for word separation  
%---------------------------------------------------      
% Help for separation (from package babel, section 22)):
% In german package the following hints are additionally available:
% "- = an explicit hyphen sign, allowing hyphenation in the rest of the word
% "| = disable ligature at this position. (e.g., Schaf"|fell)
% "~ = for a compound word mark without a breakpoint (e.g., bergauf und "~ab)
% "= = for a compound word mark with a breakpoint, allowing hyphenation in the composing words
% "" = like "-, but producing no hyphen sign (e.g., und/""oder)
%
% Describe separation hints here:
\hyphenation{
% Pro-to-koll-in-stan-zen
% Ma-na-ge-ment  Netz-werk-ele-men-ten
% Netz-werk Netz-werk-re-ser-vie-rung
% Netz-werk-adap-ter Fein-ju-stier-ung
% Da-ten-strom-spe-zi-fi-ka-tion Pa-ket-rumpf
% Kon-troll-in-stanz
}

%---------------------------------------------------      
%----- Settings for title page 
%---------------------------------------------------

\begin{titlepage}

\title{\includegraphics[width=0.6\textwidth]{pics/FU_logo.pdf}\\
{\small <Bachelorarbeit> am Institut für Informatik der Freien Universität Berlin}\\
{\small AG Robotik}\\
[6ex]
{\LARGE Untersuchung der Effizienz des Algorithmus RRT* bei der Pfadplanung autonomer Autos}\\
{\normalsize-- Exposé --}}

\author{
{\emph{\normalsize Bernd Sahre}}\\
{\normalsize Matrikelnummer: 4866829}\\
{\normalsize besahre@zedat.fu-berlin.de}\\\\
{\normalsize Betreuerin: Prof. Dr. Daniel Göhring}
}

\date{\normalsize Berlin, <Datum>}

\end{titlepage}

%%%%%%%%%%%%%%%%%%%%%%%%%%%%%%%%%%%%%%%%%%%%%%%%%%%%%%
% The content part of the documentent starts here! %%
%%%%%%%%%%%%%%%%%%%%%%%%%%%%%%%%%%%%%%%%%%%%%%%%%%%%%%

\begin{document}

\maketitle 

\thispagestyle{empty}  % eliminate page number on the title page

\blankpage

%---------------------------------------------------      
%----- Content part  
%---------------------------------------------------

\setcounter{page}{1} % page number is set to "1" otherwise it would be "3"

\section{Struktur des Exposé}
Im Folgenden habe ich Ihnen eine generelle Struktur für ein Exposé vorgegeben. Jeder Abschnitt ist mit einer Frage, welchen Inhalt dieses Kapitel abdecken sollte eingeleitet und enthält einige Erläuterungen. Bitte beachten Sie, dass das Layout dieser Vorlage doppelseitig angelegt ist.

\subsection{Motivation der Arbeit}
In welchem Bereich befindet sich die Arbeit? \\
Erläutern Sie kurz, in welchem Themenbereich Ihre Arbeit angesiedelt ist. Wo werden Sie einen Beitrag leisten? \\
\begin{itemize}
	\item  Die Arbeit ist im Bereich Robotik, Autonome Autos, Informatik angesiedelt
	
	\item Themenbereich Robotik, MidLevelPlaner, d.h. befasst sich mit der Pfadplanung autonomer Autos, also das Berechnen einer fahrbaren, möglichst optimalen Trajektorie anhand bestimmter Eingangsparameter wie Zielregion, Hindernisse, Position etc. Nach erfolgreicher Anwendung des Algorithmus soll dieser auch mit beweglichen Hindernissen umgehen können.
\end{itemize} 

\subsection{Thematische Einordnung der Arbeit}
 Welche Artikel/Literatur sind/ist relevant für diese Arbeit? \\
 Bitte geben Sie die relevanten Inhalte der Artikel kurz wieder.
\begin{itemize}
	\item Rapidly-Exploring Random Trees: A New Tool for Path Planning \cite{Lav98}
	\begin{itemize}
		\item Einführung und Erläuterung von RRT
	\end{itemize}	
	\item Incremental Sampling-based Algorithms for Optimal Motion Planning \cite{KaFra10}
	\begin{itemize}
		\item Einführung von RRT* als Verbesserung von RRT
	\end{itemize}
	\item RRT-Star-Smart: Rapid convergence implementation of RRT-Star towards optimal solution \cite{IsMaNa12}
	\begin{itemize}		
		\item Einführung von RRT*-Smart, welches schneller und besser am Optimum ist als RRT*
	\end{itemize}
	\item Optimal Path Planning using RRT* based Approaches: A Survey and Future Directions \cite{NoKhaHa16}
	\begin{itemize}
		\item Auflistung verschiedener Anwendungen von RRT, relevant besonders bei der Themenfindung, Überblick über verschiedene RRT-Arten
		\item Alle sinnvollen Optionen zur Geschwindigkeitsoptimierungen sollten genutzt werden. Welche das sind, ist noch auszuwerten.
	\end{itemize}
	\item Das Ausarbeiten von ausgewählter Literatur bzw. verwandten Arbeiten hilft Ihnen, Ihre Ziele im folgenden Abschnitt zu definieren. Daher ist eine Auseinandersetzung mit der Literatur von Beginn an notwendig, wenn es zu diesem Zeitpunkt noch nicht erschöpfend sein muss.
\end{itemize}

\subsection{Problem- und Aufgabenbeschreibung} 
Welche Ziele werden mit der Arbeit verfolgt? Und welche zentralen Fragen lassen sich daraus ableiten?
	
	\begin{enumerate}
	
		\item Eine Beschreibung des größeren Zusammenhangs, in dem das Dokument angesiedelt ist:
		\begin{itemize}
		 \item Forschung im Bereich Pfadplanung autonomer Autos, speziell für RRT.
		\end{itemize}
				
		\item  Die Beschreibung des konkreten Problems, das im Dokument behandelt wird:
		\begin{itemize}
		\item  Ist die Verwendung von RRT* sinnvoll unter Berücksichtigungvon Schnelligkeit und Genauigkeit? 
		\item Lohnt sich weitere Forschung in diese Richtung?
		\end{itemize}		

		\item Die Charakterisierung der Ziele, die das Dokument erreichen soll (z. B. der Information, die es liefern soll):
		\begin{itemize}
		\item Implementierung eines RRT*-Algorithmus zwecks Befahrung einer vorgegeben Strecke
		\item Fahren auf der Strecke und umfahren von Hindernissen
		\item Vergleich mit anderen Algorithmen: RRT, RRT* -Smart, ?
		\item Kosten/Nutzen Effizienz einer schlechteren/sehr einfachen Heuristik, welche die Laufzeit stark beschleunigt
		\end{itemize}
	
		\item Eine Begründung warum und für wen das Dokument wichtig ist:
		\begin{itemize}
		\item Einerseits für die Arbeitsgruppe Robotics, denn dadurch wird optimalerweise ein möglicher Nutzen oder ein Anwendungsgebiet für RRT* erkannt.
		\item Auch die Erkenntnis das und warum RRT* ungeeignet ist kann wertvoll sein. 				\item Andere Arbeitsgruppen oder Studenten die in diese Richtung forschen können auch davon profitieren.
		\end{itemize}
		
		\item Die Charakterisierung des Vorwissens der angepeilten Leserschaft:
		\begin{itemize}
		\item Die Hauptzielgruppe kenn sich mit allgemeinen informatischen Begriffen wie "Algorithmus", "Variable" oder "Klasse" aus,
		\item hat aber kein tieferes Verständnis in Bereichen wie RRT.
		\item  Pseudocode ist von der Zielgruppe lesbar. 
		\item Auch interessierte Kommolitonen sollten den Text verstehen können.
		\end{itemize}
		
		\item Eine Auflistung relevanter Randbedingungen, Zeitbeschränkungen, Umfangsbeschränkungen. technische Randbedingungen (Medien etc.), äußere Vorgaben (Standards) für Stil, Organisation oder Format:\\
		\begin{itemize}
		\item Abgabetermin: 22.05.2018
		\item Korrekturlesen: 10.05.2018
		\item technische Randbedingungen: Verwenden des Codes an einem Modellauto, Ausführung mindestens viermal pro Sekunde (4 Hz), unfallfreies Fahren
		\item Keine offiziellen Standards, sollte sich anderen Bachelorarbeiten angleichen
		\item Organisation: Größtenteils selbst organisiert, d.h. Statustreffen etc. muss ich mir selbst organisieren und mich drum kümmern -> Du darfst dich nie fallen lassen.		
		\end{itemize}
	\end{enumerate}

\subsection{Geplante Vorgehensweise}
Welche einzelnen Aktivitäten müssen umgesetzt werden, um die Fragen zu beantworten und das Ziel der Arbeit zu erreichen? \\
Aus den Fragen (vorheriger Abschnitt) können Sie dann gut Aktivitäten ableiten, die Ihnen helfen, Ihre weitere Arbeit zu strukturieren.
\begin{itemize}
	\item Code in C++ schreiben, welche den RRT* Algorithmus umsetzt
	\item Code aufs Auto transformieren, mit geeigneten Input/Output Parametern.
	\item genaue Mess- und Vergleichskriterien definieren
	\item Daten erheben, Algorithmen im Auto ausführen, validieren und auswerten
	\item Ergebnisse zusammenfassen und bewerten, Paper schreiben
\end{itemize}

\subsection{Technische Umsetzung}
Mit welchen softwaretechnischen Hilfsmitteln soll die Arbeit realisiert werden? \\
Selbstverständlich können Sie an der Stelle noch nicht alles wissen, aber Sie sollen sich hier bereits einen guten Überblick verschaffen.
\begin{itemize}
	\item Entwicklungsumgebung: Eclipse
	\item Programmiersprache: C++
	\item Framework: ROSCORE, roscpp
	\item Projektplanung: Kanbanboard(trello) für longterm/midterm planning
	\item Projektplanung: Evernot für short-term planning
\end{itemize}

\subsection{Erster Terminplan}
 Wie ist der generelle Zeitplan der Arbeit? \\
 Sie sollten bereits wissen, wann Sie fertig sein wollen und von dort mit der Rückwärtsterminierung starten.\\
\begin{itemize}
	\item Abgabe 22.5., KW21
	\item Fertig zum Korrekturlesen: 10.5., KW19
	\item Fertig alle implementierungen: 3.5. KW18
	\item Fertig mit allen Tutorials: KW11
\end{itemize}

\section{Konventionen zum Schreiben}
Hier werden alle Konventionen zur Entscheidung von Zweifelsfällen sprachlicher, inhaltlicher und textstruktureller Art aufgelistet.
\begin{enumerate}
	\item Sprachliche Konventionen
	\begin{itemize}
	\item Stimmen Rechtschreibung, Zeichensetzung, Grammatik?
	\item Ist die Wortwahl angemessen?
	\item Sind die Sätze verständlich (Länge, Aufbau) und angenehm lesbar?
	\end{itemize}
	\item Inhaltliche Konventionen
	\begin{itemize}
	\item Dito
	\end{itemize}
	\item Textstrukturelle Konventionen
	\begin{itemize}
	\item Absätze werden mit einer Leerzeile getrennt.
	\end{itemize}
\end{enumerate}

%---------------------------------------------------
%----- Bibliography
%---------------------------------------------------
	\bibliographystyle{natdin}
	\bibliography{references}
\end{document}