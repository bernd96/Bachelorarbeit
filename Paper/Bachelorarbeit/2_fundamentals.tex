% !TeX encoding = UTF-8
\section{Grundlagen}
Dieses Kapitel führt die verwendeten Algorithmen und Berechnungen ein.

\subsection{RRT}
Der grundlegende, hier verwendete Algorithmus ist RRT. Dieser wurde von Steven LaValle schon vor 20 Jahren, 1998, eingeführt [TODO Literaturverweis]. Die Probleme der damaligen Algorithmen war die fehlende Allgemeingültigkeit, oft lösten sie nur sehr spezifische Probleme und waren nicht auf andere Probleme übertragbar [TODO Zitat/Beleg]. Eine weitere Einschränkung war die fehlende Skalierbarkeit vieler Algorithmen, die zu kompliziert wurden, sobald die Systeme komplexer wurden. Beispiele dafür sind \textit{dynamsche Umgebungen} und \textit{nonholonomic Robots}.\\
RRT hingegen ist einfach gehalten und daher skalierbar [TODO Beleg], der Algorithmus kann direkt auf \textit{nonholonomic} und \textit{kinodynamic planning} angewendet werden.
[TODO in Quellcode formatieren]
BAUE_RRT(x_init, K, delta t)


\subsection{RRT*}

\subsection{Abstandsmetrik und Gültigkeit}
\subsection{Weitere wichtige Grundlagen}
Doppellung mit Glossar/Erklärung zu Fachwörtern?



Überleitung zum nächsten Kapitel