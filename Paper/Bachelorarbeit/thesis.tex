% !TeX encoding = UTF-8

%===============================================================================
% Font options are:
%   plain (default), serif (uses Palladio), sans-serif (uses Paratype Sans)
% Layout options are:
%   article (default, no chapters), book (for longer texts, offers \chapter)
% Paragraph options are:
%   noparskip (default, no spacing between paragraphs), parskip (spaced)
\documentclass[serif,article,noparskip]{agse-thesis}

% Global parameters, replace with actual values.
\newcommand{\thesisTitle}{Untersuchung der Effizienz von RRT* bei autonomen Autos}
% -> You may use \par (but not \\) to format the title. If you do so, you'll
%    need to manually set the 'pdftitle' attribute below.
\newcommand{\studentName}{Bernd Sahre}
%===============================================================================

\hypersetup{pdftitle={\thesisTitle}}
\hypersetup{pdfauthor={\studentName}}




\usepackage{tabularx}
\usepackage{longtable}
\usepackage{ltablex}

	\begin{document}

\coverpage[
    student/id=4866892,
    student/mail=besahre@zedat.fu-berlin.de,
    thesis/type=Bachelorarbeit,            % optional, default: Bachelorarbeit
    thesis/group={Arbeitsgruppe Robotik},
                                           % optional, default: AGSE
    thesis/advisor={Prof. Dr. Daniel Göhring},           % optional
    thesis/examiner={Prof. Dr. Daniel Göhring},
    thesis/examiner/2={Prof. Dr. Raul Rojas}, % optional
    thesis/date=\today,                    % optional, default: \today
   %title/size=\LARGE,      % set this value to overwrite automatic font size
   abstract/separate       % toggle this to move the abstract to its own page
]
{ 
Diese Arbeit beschäftigt sich mit der Verwendung des RRT* Algorithmus auf einem Modellauto. In der Bachelorarbeit von David Goedicke \citep{Goedicke18} wurden die Algorithmen $RRT^X$ und $RRT^*$ zur Berechnung eines abfahrbaren Pfades für ein Modellauto verwendet. Beide Algorithmen waren dazu in der Lage, jedoch war die Berechnungszeit zu hoch für eine Echtzeitanwendung. Dies lag unter anderem auch an den verwendeten Dubin curves und Reeds Shepp curves, die kompliziert zu berechnen sind. In dieser Arbeit wird anstelle der Reed Shepps Curves untersucht, ob es möglich ist, mit nur einer Lenkeinstellung direkt von Knoten zu Knoten zu fahren. Vier unterschiedliche Ansätze werden vorgestellt, die alle erfolglos versuchen, das Problem zu lösen.
}

\include{declaration}

\cleardoublepage

\tableofcontents

\cleardoublepage

\mainmatter

% !TeX encoding = UTF-8
\section{Einleitung}
Schon die Griechen der Antike träumten von autonomen, selbstfahrenden Fahrzeugen [TODO Literaturverweis]. Mittlerweile ist die Forschung in diesen Bereichen so weit fortgeschritten, dass dieser Traum schon bald Wirklichkeit werden könnte. Dabei werden unterschiedliche Ansätze verfolgt, das Auto sicher durch den Straßenverkehr zu bringen. Dabei heißt sicher, dass das Auto während der Fahrt weder sich noch andere Verkehrsteilnehmer gefährdet. Neben der Sicherheit ist jedoch auch das Erreichen des Ziels wichtig, bei dem das Auto seinen Weg durch eine Umgebung mit statischen und dynamischen, das heißt sich bewegenden Hindernissen finden muss. Dies wird durch einen Pfadplaner gewährleistet, der -informell gesprochen - aus der eigenen Position, dem Zielbereich und unter Berücksichtigung aller statischen und sich bewegenden Hindernissen einen sicheren Pfad zum Ziel findet. Dieser Pfad ist dann durch das Auto abfahrbar.\\
Um diese Aufgabe zu meistern, existieren unterschiedliche Ansätze, um dem Auto je nach Anwendungsfall bei gegebenen Ziel eine \textit{Trajektorie} vorzuschlagen. Die jeweiligen Algorithmen unterscheiden sich in Ausführungszeit, Genauigkeit, Sicherheit und berechnen unterschiedlich optimale Pfade.  \\
Das Dahlem Center for Machine Learning and Robotics untersucht maschinelles Lernen und Anwendungen intelligenter Systeme. Dazu haben sich vier Arbeitsgruppen der Freien Universität Berlin zusammengeschlossen:
\begin{itemize}
\item Intelligent Systems and Robotics (Prof. Dr. Raúl Rojas)
\item Autonomos Cars (Prof. Dr. Daniel Göhring)
\item Artifical and Collective Intelligence (Prof. Dr. Tim Landgraf)
\item Logic and automatic proofs (Christoph Benzmüller)
\end{itemize}
Ein Forschungsgebiet ist die Entwicklung und Analyse autonomer Autos. Zur Pfadplanung wird hauptsächlich das Prinzip elastischer Bänder (Time-Elastic-Bands, \citep[vgl.][]{RoeHoBe}) zur Erzeugung von \textit{Trajektorien} benutzt. Doch auch die Untersuchung und Analyse anderer Algorithmen ist interessant, um zu überprüfen ob sich eine vertiefte Forschung in diesen Bereichen lohnt. \\ 
Diese Bachelorarbeit untersucht einen dieser Algorithmen zur Pfadplanung, \textit{RRT*}, auf seine Tauglichkeit, über eine vorgegebene Fahrbahn mit Hindernissen eine abfahrbare, sichere und möglichst optimale \textit{Trajektorie} zu finden. 

\subsection{Aufbau der Arbeit}
(Arbeitstitel)
[TODO doppelung mit struktur]  \\ 
\subsubsection{Problemanalyse}
Diese Arbeit untersucht die Effizienz des\textit{ RRT*}-Algorithmus unter Anwendung bei autonomen Autos. Dazu fährt das Auto eine vorgegebene Strecke ab [TODO Bild einfügen]. Auf dieser Strecke werden erst statische, dann sich bewegende Hindernisse platziert. \\
Das Auto soll diese Strecke in möglichst kurzer Zeit mit einem möglichst optimalen Pfad abfahren, ohne eines dieser Hindernisse zu berühren. Dazu sollte das Auto den Algorithmus idealerweise 30 Mal pro Sekunde ausführen, mindestens aber vier Mal pro Sekunde. \\
Die Effizienz, Sicherheit und Effektivität des Algorithmus wird bei verschiedenen Eingabeparametern untersucht und bewertet. \\

\subsubsection{Anmerkung zur Gestaltung der Arbeit}
[TODO Plagiat mit Prof abklären (ist ne 1:1 Kopie)]
Für die im Folgenden verwendeten personenbezogenen
Ausdrücke wurde, um die Lesbarkeit der Arbeit zu erhöhen,
ausschließlich die männliche Schreibweise gewählt. Desweiteren werden eine
Reihe von englischen Bezeichnungen und Fachwörtern verwendet, um einerseits dem
interessierten Leser das Studium der fast ausschließlich englischen
Fachliteratur zu erleichtern und andererseits bestehende Fachbegriffe nicht durch die Übersetzung zu verfälschen. Diese Begriffe
wurden im Gegensatz zum restlichen Text in kursiver Schrift formatiert.

\subsubsection{Glossar}
\begin{tabularx}{\textwidth}{l|X}
 \textbf{Begriff}  & \textbf{Erklärung}  \\
\hline Trajektorie & Die Strecke, die dem Low-Level-Planer des Autos übergeben wird\\
Low-Level-Planer & Steuert direkt die Motoren des Autos, Lenkung und Antrieb, um eine vorgegebene Trajektorie möglichst genau abzufahren. \\
RRT & Rapidly-Exploring Random Tree \citep{Lav98}, ein Algorithmus zur Findung eines Pfades zum Ziel durch unbekanntes Gelände, wird bei [Querverweis] noch weiter erläutert\\
RRT* & Eine verbesserte Variante des RRT, bei dem die Pfade optimiert werden. Asymptoptisch optimal \citep{Bernd} \\
ROS & Robot Operating Systems, eine Open-Source Sammlung an Software-Bibliotheken und Werkzeugen zur 	Kreation von Anwendungen zur Robotik. \\
Lidar & Light detection and ranging; Radarscanner am Auto, für Abstandsmessung zu Hindernissen.\\
Nonholonomic Robots & Roboter, die gewissen kinematischen Einschränkungen unterworfen sind. Ein Auto zum Beispiel kann sich nicht in jede beliebige Richtung bewegen, sondern ist z.B. durch den maximalen Lenkwinkel und den Wenderadius eingeschränkt und kann nicht jeden beliebigen Punkt sofort, mit nur einem Schritt, erreichen.  \\
Dynamische Umgebung & Eine Umgebung mit dynamischen, also sich bewegenden oder veränderlichen Hindernissen. \\
Kinodynamic planning & Beschreibt eine Klasse von Problemen bei der physikalische Einschränkungen wie Geschwindigkeit, Beschleunigung und Kräfte zusammen mit kinematischen Einschränkungen (Hindernisvermeidung) berücksichtigt werden müssen. \\
hochdimensionale Probleme & [TODO]\\
randomisierte Algorithmen & [TODO]\\
Kinematik & \\
Odroid & \\
Arduino Nano & \\
Gyroskop & \\
Odometrie & \\
Remote Procedure Call & \\

\end{tabularx} 
\subsubsection{Struktur}
[TODO Querverweise]
Nach einer kurzen Hinführung zum Thema werden zuallererst die Grundlagen besprochen, um das Verständnis der nachfolgenden Kapitel zu erleichtern. Danach wird die Umsetzung des Algorithmus auf das Auto sowie die technischen Details und Hintergründe beschrieben. Im vierten Kapitel werden die Ergebnisse der Testfahrten vorgestellt und bewertet. Zum Schluss werden in einem Fazit alle Schlussfolgerungen nochmals zusammengefasst und ein Ausblick auf weitere mögliche Forschungsmöglichkeiten gegeben.

\subsubsection{Wichtige Quellen und deren Beitrag zum Thema}
Es existieren[TODO RRT-Quellen \citep[vgl][]{Lav98}, RRT*-Quellen, Übersichtsliteratur]
\\
Nun werden wir uns den wissenschaftlichen Grundlagen der Arbeit widmen.


% !TeX encoding = UTF-8
\section{Grundlagen}
Dieses Kapitel führt die verwendeten Algorithmen und Berechnungen ein.

\subsection{RRT}
Der grundlegende, hier verwendete Algorithmus ist RRT. Dieser wurde von Steven LaValle schon vor 20 Jahren, 1998, eingeführt [TODO Literaturverweis]. Die Probleme der damaligen Algorithmen war die fehlende Allgemeingültigkeit, oft lösten sie nur sehr spezifische Probleme und waren nicht auf andere Probleme übertragbar [TODO Zitat/Beleg]. Eine weitere Einschränkung war die fehlende Skalierbarkeit vieler Algorithmen, die zu kompliziert wurden, sobald die Systeme komplexer wurden. Beispiele dafür sind \textit{dynamsche Umgebungen} und \textit{nonholonomic Robots}.\\
RRT hingegen ist einfach gehalten und daher skalierbar [TODO Beleg], der Algorithmus kann direkt auf \textit{nonholonomic} und \textit{kinodynamic planning} angewendet werden.
[TODO in Quellcode formatieren]
BAUE_RRT(x_init, K, delta t)


\subsection{RRT*}

\subsection{Abstandsmetrik und Gültigkeit}
\subsection{Weitere wichtige Grundlagen}
Doppellung mit Glossar/Erklärung zu Fachwörtern?



Überleitung zum nächsten Kapitel
% !TeX encoding = UTF-8
\section{Umsetzung}
Bevor wir uns den notwendigen Anpassungen des \textit{RRT*}-Algorithmus widmen können, müssen wir die kinematischen und physikalischen Beschränkungen des Autos analysieren. Anschließend wird das verwendete Framework ROS - Robot Operating Systems - vorgestellt, bevor wir uns mit der Wahl einer geeigneten Metrik und Kostenfunktion beschäftigen.
\subsection{Hardwareausstattung der Autos}
Das Dahlem Center for Machine Learning and Robotics arbeitet mittlerweile mit dem Modelfahrzeug "AutoNOMOS Mini v3" (1:10). Der Hauptcomputer auf dem Auto ist ein \textit{Odroid}(XU4 64GB) mit Ubuntu Linux als Betriebssystem und ROS (Robot Operation Systems) als Steuerungssystem \citep{fubAuto}. \\ [TODO Bild einfügen]
Motorisiert ist das Auto mit einem bürstenlosen [TODO??] DC-Servomotor FAULHABER 2232. Die Lenkung wird von dem Servomotor HS-645-MG übernommen, beide Motoren werden mithilfe einer \textit{Arduino Nano} Platine gesteuert. \\
Zur Wahrnehmung der Umgebung besitzt das AutoNOMOS Mini v3 mit dem RPLIDAR A2 360 einen rotierenden Laserscanner, der in der Lage ist, die Umgebung des Autos auf Hindernisse zu überprüfen. Als Rückgabewert liefert der RPLIDAR pro Gradwinkel den Wert, wie weit das nächste Hindernis in dieser Richtung entfernt ist, also insgesamt 360 Werte (einen pro Winkel). \\
Auf dem oberen Teil des Autos befestigt ist das \textit{Kinect-type stereoscopic system} (Intel RealSense SR300), welches eine Wolke aus 3D Punkten liefert, die dazu benutzt werden kann, Hindernisse zu erkennen. Außerdem kann die Kamera des \textit{Kinect-type} Sensors dazu benutzt werden, Fahrbahnmarkierungen und Objekte direkt vor dem Auto zu lokalisieren.\\
Der letzte äußere Sensor, auch am oberen Teil des Autos angebracht, ist die Fischaugen-Kamera. Diese zeigt nach oben, zur Decke, und kann dazu benutzt werden bestimmte markante, feststehende Objekte zu lokalisieren, damit das AutoNOMOS Mini v3 sich auch innerhalb von Räumen orientieren kann. Dazu kann eine GPS Navigationseinheit simuliert werden, indem die an der Decke angebrachten vier Lampen in unterschiedlichen Farben leuchten. \\
Die Sensoren sind entweder via USB 3.0 an der Hauptplatine oder direkt am \textit{Odroid} angeschlossen. \\
An inneren Sensoren besitzt das AutoNOMOS Mini v3 eine MPU6050, die einen Beschleunigungssensor und ein \textit{Gyroskop} enthält. Mithilfe dieser MPU kann das AutoNOMOS Mini v3 seine Orientierung, seine Richtung im Raum bestimmen. Außerdem können Messungen zur \textit{Odometrie} ergänzt werden.\\
Das AutoNOMOS Mini v3 wird über eine 14,8 V Batterie mit Energie versorgt.

\subsection{Software: ROS - Robot Operating Systems}
ROS stellt Bibliotheken und Werkzeuge zur Verfügung, die Software-Entwicklern helfen sollen, Robotik Anwendungen zu kreieren \citep{ROS}. Unter anderem beinhaltet ROS Gerätetreiber, Bibliotheken, Visualisierungswerkzeuge, Paketmanagement und vieles mehr. ROS ist Open Source und unter der BSD Lizenz verfügbar.
\subsubsection{Architektur}
Mithilfe von ROS können so genannte \textit{Nodes}, ausführbare Programme, erzeugt werden, die über so genannte \textit{Topics} kommunizieren können. Dies passiert über einen anonymisierten Publisher/Subscriber Mechanismus, das heißt Daten generierende Knoten können auf relevanten \textit{Topics} Nachrichten senden, und interessierte Knoten können von relevanten \textit{Topics} Nachrichten empfangen. \\
Für jedes \textit{Topic} ist dabei auch die Nachrichtenart definiert, die für dieses \textit{Topic} veröffentlicht und von diesem \textit{Topic} empfangen werden. Dies können neben simplen Datentypen auch komplexe, selbst definierte Datenstrukturen sein. Dabei wird nur dieser eine, vorher festgelegte Datentyp der Nachricht vom \textit{Topic} akzeptiert.
\textit{Topics} stellen nur eine unidirektionale Verbindung zur Verfügung. Für die Abwicklung von zum Beispiel \textit{Remote Procedure Calls} sind sogenannte Services zuständig. Diese ermögliche, eine Antwort auf eine bestimmte Anfrage nach dem Client Server Prinzip zurückzusenden.
\subsection{APIs und Einbettung zu bereits vorhandene Knoten}
Das Dahlem Center for Machine Learning and Robotics entwickelte ROS-Pakete für die Steuerung autonomer Autos. Diese Pakete und daraus resultierenden ROS-Nodes können dazu genutzt werden, den von mir entwickelten RRT*-Pfadplaner möglichst gut einzubetten. So kann durch das visuelle indoor GPS die Position des Autos bestimmt werden, die der Pfadplaner für seine Berechnungen braucht. Die resultierende \textit{Trajektorie}, die der Pfadplaner entwickelt, wird einem Steuerungsknoten übergeben, der diese \textit{Trajektorie} in Motorbefehle, also Beschleunigungen und Lenkungen, umsetzt. 
\subsubsection{Bestimmung der Odometry und visual GPS}
[TODO was benutzte ich jetzt?]
\subsubsection{Steuerungsknoten fubMigController}
Dieser Knoten lauscht auf das \textit{Topic} \verb|"planned_path"|. Auf \verb|"planned_path"| kann eine \textit{Trajektorie} publiziert werden, die dann mithilfe des Steuerungsknotens vom Auto abgefahren wird. Dabei kümmert sich dieser Steuerungsknoten allerdings nicht um etwagige Hindernisse, die mit dem Auto kollidieren könnten. Somit muss der Pfadplaner selbst alle Kollisionen mit Hindernissen ausschließen. \\
Das Format der \textit{Trajektorie} wurde von der Arbeitsgruppe der FU Berlin definiert und besteht aus
\begin{itemize}
\item \verb|std_msgs/Header|: Hier wird die aktuelle Zeit gespeichert.
\item \verb|string child_frame_id|: [TODO]
\item \verb|fub_trajectory_msgs/TrajectoryPoint[] trajectory|: Eine Liste aus Trajektorie-Punkten.
\end{itemize}
Ein Trajektorien-Punkt symbolisiert einen abzufahrenden Knotenpunkt und wiederum besteht aus
\begin{itemize}
\item \verb|geometry_msgs/Pose pose|: Hier sind Position und Orientierung des Autos gespeichert.
\item \verb|geometry_msgs/Twist velocity|:Hier wird die Geschwindigkeit des Autos an diesem Punkt gespeichert.
\item \verb|geometry_msgs/Twist acceleration| Hier wird die Beschleunigung des Autos an diesem Punkt gespeichert.
\end{itemize}
Die Position wird in x-Position und y-Position angegeben, ausgehend von einer Ecke des Raumes. Die Orientierung wird durch ein \textit{Quaternion} dargestellt, bei dem durch vier Werte die Drehung in jeder Richtung des Raumes genau bestimmt ist. Allerdings genügt uns die Drehung um die z-Achse, also die Drehrichtung über die Vertikalachse, weshalb dieses Quaternion in einen Winkel, der sogenannten Gierung, umgerechnet wird.
\\
Nachdem die notwendige Infrastruktur erläutert wurde, folgt eine kurze Betrachtung der kinematischen und physikalischen Einschränkungen, denen das AutoNOMOS Mini v3 unterworfen ist. Anschließend können wir uns endlich dem Kernstück der Arbeit, dem eigentlichen RRT*-Pfadplaner, widmen.

\subsection{Kinematische und physikalische Einschränkungen}


\subsection{RRT*-Pfadplaner}

\subsubsection{RRT* }
Dazu sind Änderungen besonders an zwei Stellen nötig: 
\begin{enumerate}
\item Es muss bei der Auswahl des Vaterknotens überprüft werden, ob der eingefügte Knoten über den Vaterknoten überhaupt erreicht werden kann.
\item Beim Rewiring - dem Neuverknüpfen der Knoten - wird nicht die alte Verbindung zum Vaterknoten gelöscht. Stattdessen wird ein neuer Knoten erzeugt, der als Vaterknoten den eingefügten Knoten \verb|x_new| hat. Dies ist nötig, weil 
\end{enumerate}

\begin{itemize}
\item  Datenstruktur Nodes
\item Berechnung der Orientierung
\item Rewiring
\end{itemize}
\subsection{Metrik und Kostenfunktion}
Bestrafen für harte Lenkung bzw. (starke) Lenkänderung
Belohnung für geradeausfahren und kurze Wege




\subsection{Dokumentation der Durchführung und entstandener Artefakte}
\subsection{Verwendete Metriken}
\subsection{Beschreibung besonderer Schwierigkeiten und wie diese umgangen wurden}
\subsection{(Evaluation - nur wenn ich dafür Zeit habe)}


\subsubsection{Tests und Testdatensätze/Szenarien für die Software)}
\subsubsection{Korrektheitsbeweise}

% !TeX encoding = UTF-8
\section{Ausblick und Fazit}
\label{sec:Zusammenfassung}
Hier kommt die Zusammenfassung aller Ergebnisse hin, Fazit, Bewertung der Arbeitsweise und der Ergebnisse, was habe ich mitgenommen, was kann ich zurückgeben

\subsection{Ausblick}
Weitere Forschungen, mit Entfernungsnorm rumspielen, z.B. Manhattan Norm anstatt Euklidische Norm: Bessere Laufzeit, Belohnt geradeausfahren und 90 Grad Turns \\
RRT*-Smart: Sampling zum ziel hin, bei Problemregionen (nahe an Hindernissen) \\
Bessere, optimierte dynamische Datenstruktur für die Knoten (Vorschläge machen, kd-tree, quadtree, ...)\\


[TODO] \\
Hier werden dann Ergebnisse bekannt gegeben, Bilder, Grafiken angefügt die den Erfolg/Misserfolg des Alg. RRT* deutlich machen.
...

\bibliography{bibliography}

%\appendix
%% !TeX encoding = UTF-8
\section{Anhang}

Quellcode der \LaTeX-Klasse \texttt{node.cpp}:\footnote{Es ist nicht üblich,
den gesamten produzierten Quellcode bei einer Abschlussarbeit in Textform
abzugeben. Werde das auch noch kürzen und nur die wichtigsten FUnktionen reinpacken}

\lstinputlisting[
    language={C++},
    basicstyle=\footnotesize\ttfamily,
    numbers=left,
    numberstyle=\footnotesize\ttfamily,
    stepnumber=5,
    inputencoding=utf8,
    extendedchars=true,
   literate={ä}{{\"a}}1 {ü}{{\"u}}1,
]{Node.cpp}


\end{document}