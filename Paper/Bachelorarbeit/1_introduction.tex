% !TeX encoding = UTF-8
\section{Einleitung}
Schon die Griechen der Antike träumten von autonomen, selbstfahrenden Fahrzeugen. Mittlerweile ist die Forschung in diesen Bereichen so weit fortgeschritten, dass dieser Traum schon bald Wirklichkeit werden könnte. Dabei werden unterschiedliche Ansätze verfolgt, das Auto sicher durch den Straßenverkehr zu bringen. Dabei heißt sicher, dass das Auto während der Fahrt weder sich noch andere Verkehrsteilnehmer gefährdet. Neben der Sicherheit ist jedoch auch das Erreichen des Ziels wichtig, bei dem das Auto seinen Weg durch eine Umgebung mit statischen und dynamischen, das heißt sich bewegenden Hindernissen finden muss. \\
Um dies zu erreichen, existieren unterschiedliche Algorithmen, die dafür zuständig sind, dem Auto bei gegebenen Ziel eine passende \textit{Trajektorie} vorzuschlagen. Die Algorithmen unterscheiden sich in Ausführungszeit, Genauigkeit und berechnen unterschiedlich optimale Pfade.  \\
Die Arbeitsgruppe Robotik der Freien Universität Berlin benutzt in ihren Forschungen hauptsächlich das Prinzip elastischer Bänder zur Erzeugung von \textit{Trajektorien} (Time-Elastic-Bands[hier Literaturverweis]).Doch auch die Untersuchung und Analyse anderer Algorithmen ist interessant, um zu überprüfen ob sich vertiefte Forschung in diesen Bereichen lohnt. \\
Diese Bachelorarbeit untersucht einen dieser Algorithmen, \textit{RRT*}, auf seine Tauglichkeit, über eine vorgegebene Fahrbahn mit Hindernissen eine abfahrbare und möglichst gute \textit{Trajektorie} zu finden.

\subsection{Bemerkung zu Fachwörtern}
Bevor wir uns der Problemanalyse und der Einarbeitung in die Grundlagen des Themas  widmen, hier noch eine kurze Anmerkungen zur Gestaltung der
vorliegenden Arbeit: \\
Für die im Folgenden verwendeten personenbezogenen
Ausdrücke wurde, um die Lesbarkeit der Arbeit zu erhöhen,
ausschließlich die männliche Schreibweise gewählt. Des Weiteren werden eine
Reihe von englischen Bezeichnungen und Fachwörtern verwendet, um einerseits dem
interessierten Leser das Studium der fast ausschließlich englischen
Fachliteratur zu erleichtern und andererseits bestehende Fachbegriffe nicht durch die Übersetzung zu verfälschen. Diese Begriffe
wurden im Gegensatz zum restlichen Text in kursive Schrift formatiert.
\begin{tabular}{•}
\textbf{Begriff} & \textbf{Erklärung} \\
Trajektorie & Die Strecke, die dem Low-Level-Planer des Autos übergeben wird\\
Low-Level-Planer & Steuert direkt die Motoren des Autos, Lenkung und Antrieb, um eine vorgegebene Trajektorie möglichst genau abzufahren. \\
RRT & Rapidly-Exploring Random Tree, ein Algorithmus zur Findung eines Pfades zum Ziel durch unbekanntes Gelände, wird bei [Querverweis] noch weiter erläutert\\
RRT* & Eine verbesserte Variante des RRT, bei dem die Pfade optimiert werden. Asymptoptisch optimal [Literaturverweis]\\
ROS & Robot Operating Systems, eine Open-Source Sammlung an Software-Bibliotheken und Werkzeugen zur 	Kreation von Anwendungen zur Robotik. \\
Lidar & Light detection and ranging; Radarscanner am Auto, für Abstandsmessung zu Hindernissen.\\


\end{tabular}
