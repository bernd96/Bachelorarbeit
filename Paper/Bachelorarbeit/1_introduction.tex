% !TeX encoding = UTF-8
\section{Einleitung}
Schon die Griechen der Antike träumten von autonomen, selbstfahrenden Fahrzeugen [TODO Literaturverweis]. Mittlerweile ist die Forschung in diesen Bereichen so weit fortgeschritten, dass dieser Traum schon bald Wirklichkeit werden könnte. Dabei werden unterschiedliche Ansätze verfolgt, das Auto sicher durch den Straßenverkehr zu bringen. Dabei heißt sicher, dass das Auto während der Fahrt weder sich noch andere Verkehrsteilnehmer gefährdet. Neben der Sicherheit ist jedoch auch das Erreichen des Ziels wichtig, bei dem das Auto seinen Weg durch eine Umgebung mit statischen und dynamischen, das heißt sich bewegenden Hindernissen finden muss. \\
Um dies zu erreichen, existieren unterschiedliche Algorithmen, die dafür zuständig sind, dem Auto bei gegebenen Ziel eine passende \textit{Trajektorie} vorzuschlagen. Die Algorithmen unterscheiden sich in Ausführungszeit, Genauigkeit und berechnen unterschiedlich optimale Pfade.  \\
Die Arbeitsgruppe Robotik der Freien Universität Berlin benutzt in ihren Forschungen hauptsächlich das Prinzip elastischer Bänder zur Erzeugung von \textit{Trajektorien} (Time-Elastic-Bands[TODO hier Literaturverweis]).Doch auch die Untersuchung und Analyse anderer Algorithmen ist interessant, um zu überprüfen ob sich vertiefte Forschung in diesen Bereichen lohnt. \\
Diese Bachelorarbeit untersucht einen dieser Algorithmen, \textit{RRT*}, auf seine Tauglichkeit, über eine vorgegebene Fahrbahn mit Hindernissen eine abfahrbare und möglichst gute \textit{Trajektorie} zu finden.



\subsection{Aufbau der Arbeit}
(Arbeitstitel) \\
\subsubsection{Problemanalyse}
Diese Arbeit untersucht die Effizienz des RRT*-Algorithmus unter Anwendung bei autonomen Autos. Dazu fährt das Auto eine vorgegebene Strecke ab [TODO Bild einfügen]. Auf dieser Strecke werden erst statische, dann sich bewegende Hindernisse platziert. \\
Das Auto soll diese Strecke in möglichst kurzer Zeit mit einem möglichst optimalen Pfad abfahren, ohne eines dieser Hindernisse zu berühren. Dazu sollte das Auto den Algorithmus idealerweise 30 Mal pro Sekunde ausführen, mindestens aber vier Mal pro Sekunde. \\
Die Effizienz, Sicherheit und Effektivität des Algorithmus wird bei verschiedenen Eingabeparametern untersucht und bewertet. \\

\subsubsection{Anmerkung zur Gestaltung der Arbeit}
Für die im Folgenden verwendeten personenbezogenen
Ausdrücke wurde, um die Lesbarkeit der Arbeit zu erhöhen,
ausschließlich die männliche Schreibweise gewählt. Des Weiteren werden eine
Reihe von englischen Bezeichnungen und Fachwörtern verwendet, um einerseits dem
interessierten Leser das Studium der fast ausschließlich englischen
Fachliteratur zu erleichtern und andererseits bestehende Fachbegriffe nicht durch die Übersetzung zu verfälschen. Diese Begriffe
wurden im Gegensatz zum restlichen Text in kursiver Schrift formatiert.

\subsubsection{Glossar}
\begin{tabular}{•}
\textbf{Begriff} & \textbf{Erklärung} \\
Trajektorie & Die Strecke, die dem Low-Level-Planer des Autos übergeben wird\\
Low-Level-Planer & Steuert direkt die Motoren des Autos, Lenkung und Antrieb, um eine vorgegebene Trajektorie möglichst genau abzufahren. \\
RRT & Rapidly-Exploring Random Tree, ein Algorithmus zur Findung eines Pfades zum Ziel durch unbekanntes Gelände, wird bei [Querverweis] noch weiter erläutert\\
RRT* & Eine verbesserte Variante des RRT, bei dem die Pfade optimiert werden. Asymptoptisch optimal [TODO Literaturverweis]\\
ROS & Robot Operating Systems, eine Open-Source Sammlung an Software-Bibliotheken und Werkzeugen zur 	Kreation von Anwendungen zur Robotik. \\
Lidar & Light detection and ranging; Radarscanner am Auto, für Abstandsmessung zu Hindernissen.\\
Nonholonomic Robots & Roboter, die gewissen kinematischen Einschränkungen unterworfen sind. Ein Auto zum Beispiel kann sich nicht in jede beliebige Richtung bewegen, sondern ist z.B. durch den maximalen Lenkwinkel und den Wenderadius eingeschränkt und kann nicht jeden beliebigen Punkt sofort, mit nur einem Schritt, erreichen.  \\
Dynamische Umgebung & Eine Umgebung mit dynamischen, also sich bewegenden oder veränderlichen Hindernissen. \\
Kinodynamic planning & Beschreibt eine Klasse von Problemen bei der physikalische Einschränkungen wie Geschwindigkeit, Beschleunigung und Kräfte zusammen mit kinematischen Einschränkungen (Hindernisvermeidung) berücksichtigt werden müssen. \\
\subsubsection{Struktur}
Nach einer kurzen Hinführung zum Thema werden zuallererst die Grundlagen besprochen, um das Verständnis der nachfolgenden Kapitel zu erleichtern. Danach wird die Umsetzung des Algorithmus auf das Auto sowie die technischen Details und Hintergründe beschrieben. Im vierten Kapitel werden die Ergebnisse der Testfahrten vorgestellt und bewertet. Zum Schluss werden in einem Fazit alle Schlussfolgerungen nochmals zusammengefasst und ein Ausblick auf weitere mögliche Forschungsmöglichkeiten gegeben.

\subsubsection{Wichtige Quellen und deren Beitrag zum Thema}
Es existieren[TODO]
\\
Nun werden wir uns den wissenschaftlichen Grundlagen der Arbeit widmen.

\end{tabular}
