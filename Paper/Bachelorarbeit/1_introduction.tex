% !TeX encoding = UTF-8
\section{Einführung}
Kaum ein Thema beherrscht Diskussionen über die Mobilität der Zukunft wie autonomes Fahren. Doch während Philosophen und Politik vor allem auch über ethische Fragen streiten (vgl. mit irgendeinem aufgegabelten Artikel über Ethikprobleme zwischen zwei Übeln zu wählen), steht die Forschung vor der Aufgabe, das Auto sicher durch den Straßenverkehr zu bringen. Dabei heißt sicher, dass das Auto während der Fahrt weder sich noch andere Verkehrsteilnehmer gefährdet. \\
Um dies zu gewährleisten, existieren unterschiedliche Algorithmen, die dafür zuständig sind, dem Auto bei gegebenen Ziel eine passende \textit{Trajektorie} vorzuschlagen. Die Algorithmen unterscheiden sich in Ausführungszeit, Genauigkeit und berechnen unterschiedlich optimale Pfade. Diese Bachelorarbeit untersucht einen dieser Algorithmen, \textit{RRT*}, auf seine Tauglichkeit, über eine vorgegebene Fahrbahn, die mit Hindernissen bestückt ist, zu fahren. \\

\subsection{Motivation}
Die Arbeitsgruppe Robotik der Freien Universität Berlin benutzt in ihren Forschungen hauptsächlich das Prinzip elastischer Bänder zur Erzeugung von \textit{Trajektorien} (Time-Elastic-Bands[hier Literaturverweis]).Doch auch die Verwendung und Probleme anderer Algorithmen ist interessant, insbesondere ob sich vertiefte Forschung in diesem Bereich lohnt. Dabei wird weniger Wert auf eine korrekte Orientierung des Autos in jedem Zustand gesetzt, sondern Hauptsache das Auto hat die Hindernisstrecke in kürzester Zeit ohne Fehler überwunden.
\subsection{Bemerkung zu Fachwörtern}
Bevor wir uns der Problemanalyse und der Einarbeitung in die Grundlagen des Themas  widmen, hier noch eine kurze Anmerkungen zur Gestaltung der
vorliegenden Arbeit: \\
Für die im Folgenden verwendeten personenbezogenen
Ausdrücke wurde, um die Lesbarkeit der Arbeit zu erhöhen,
ausschließlich die männliche Schreibweise gewählt. Des Weiteren werden eine
Reihe von englischen Bezeichnungen und Fachwörtern verwendet, um einerseits dem
interessierten Leser das Studium der fast ausschließlich englischen
Fachliteratur zu erleichtern und andererseits bestehende Fachbegriffe nicht durch die Übersetzung zu verfälschen. Diese Begriffe
wurden im Gegensatz zum restlichen Text in kursive Schrift formatiert.
